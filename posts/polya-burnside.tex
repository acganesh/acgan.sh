%% ================================================================================
%% This LaTeX file was created by AbiWord.                                         
%% AbiWord is a free, Open Source word processor.                                  
%% More information about AbiWord is available at http://www.abisource.com/        
%% ================================================================================

\documentclass[a4paper,portrait,12pt]{article}
\usepackage[latin1]{inputenc}
\usepackage{calc}
\usepackage{setspace}
\usepackage{fixltx2e}
\usepackage{graphicx}
\usepackage{multicol}
\usepackage[normalem]{ulem}
%% Please revise the following command, if your babel
%% package does not support en-US
\usepackage[en]{babel}
\usepackage{color}
\usepackage{hyperref}
 
\begin{document}


\begin{flushleft}
P\'{o}lya-Burnside Enumeration in Combinatorics
\end{flushleft}


\begin{flushleft}
Adithya Ganesh
\end{flushleft}





1





\begin{flushleft}
A Class of Problems
\end{flushleft}





\begin{flushleft}
The following problem in chemistry is historically significant, as G. P\'{o}lya originally popularized his theory
\end{flushleft}


\begin{flushleft}
through applications in chemical enumeration. How many different chemical compounds can be made by
\end{flushleft}


\begin{flushleft}
attaching H, CH3 , or OH radicals to each of the carbon atoms in the benzene ring pictured below?
\end{flushleft}


\begin{flushleft}
C
\end{flushleft}





\begin{flushleft}
C
\end{flushleft}





\begin{flushleft}
C
\end{flushleft}





\begin{flushleft}
C
\end{flushleft}





\begin{flushleft}
C
\end{flushleft}





\begin{flushleft}
C
\end{flushleft}





\begin{flushleft}
G. P\'{o}lya, R. C. Read (1987). Combinatorial Enumeration of Groups, Graphs, and Chemical Compounds. New York: Springer-Verlag.
\end{flushleft}





\begin{flushleft}
Here are other problems that can be approached using P\'{o}lya-Burnside.
\end{flushleft}


\begin{flushleft}
1. In how many ways can an n × n tablecloth be colored with k colors?
\end{flushleft}


\begin{flushleft}
2. How many different necklaces can be made with n beads and k colors?
\end{flushleft}


\begin{flushleft}
3. How many ways can the faces of a polyhedron be colored using at most n colors?
\end{flushleft}


\begin{flushleft}
4. Find the number of simple graphs with n vertices, up to isomorphism.
\end{flushleft}


\begin{flushleft}
One can observe a common theme of enumerating the number of objects with some equivalence under
\end{flushleft}


\begin{flushleft}
symmetry.
\end{flushleft}





2





\begin{flushleft}
An Example: Coloring a Flag
\end{flushleft}





\begin{flushleft}
Problem:
\end{flushleft}


\begin{flushleft}
How many ways are there to color a flag with n stripes lined side by side with k colors?
\end{flushleft}


···


\begin{flushleft}
n stripes, k colors
\end{flushleft}





\begin{flushleft}
Do not count as different flags with colors {``}flipped.'' The following two flags would be considered the same:
\end{flushleft}





1





\newpage
2.1





\begin{flushleft}
Solving it with Standard Methods
\end{flushleft}





\begin{flushleft}
Let's take the simple case when n = 4 and k = 2.
\end{flushleft}


\begin{flushleft}
Assume we count the number of patterns normally, without accounting for reflection. N = 24 . Let Nr denote
\end{flushleft}


4


\begin{flushleft}
the number of distinct colorings under reflection. Nr 6= 22 , as one might think! We need to separately handle
\end{flushleft}


\begin{flushleft}
symmetric patterns and asymmetric patterns.
\end{flushleft}


\begin{flushleft}
An asymmetric pattern like
\end{flushleft}





\begin{flushleft}
yields a new pattern that we don't want to double count ,
\end{flushleft}





\begin{flushleft}
, when it is reflected. Should be divided by 2.
\end{flushleft}


\begin{flushleft}
A symmetric pattern like
\end{flushleft}


\begin{flushleft}
divide by 2 here.
\end{flushleft}





\begin{flushleft}
when reflected does not create a new pattern. We don't need to
\end{flushleft}





\begin{flushleft}
Let A be the number of asymmetric patterns not accounting for reflection, and S be the number of symmetric
\end{flushleft}


\begin{flushleft}
patterns. (Note that A + S = 24 .)
\end{flushleft}


\begin{flushleft}
The number of patternings accounting for reflection, Nr , is given by
\end{flushleft}





\begin{flushleft}
Nr =
\end{flushleft}





\begin{flushleft}
24 $-$ S
\end{flushleft}


\begin{flushleft}
A
\end{flushleft}


\begin{flushleft}
+S =
\end{flushleft}


\begin{flushleft}
+S
\end{flushleft}


2


2





\begin{flushleft}
S = 22 , since picking the first two squares uniquely defines the last two. Hence Nr =
\end{flushleft}


\begin{flushleft}
Exercise. Show that in the general case, Nr =
\end{flushleft}





2.2





24 $-$22


2





+ 22 = 10.





\begin{flushleft}
kn +kb(n+1)/2c
\end{flushleft}


.


2





\begin{flushleft}
Applying P\'{o}lya's Theory
\end{flushleft}





\begin{flushleft}
We will first apply P\'{o}lya's method without explaining how it works.
\end{flushleft}


\begin{flushleft}
Solution.
\end{flushleft}





\begin{flushleft}
Cycle Notation
\end{flushleft}


?


\begin{flushleft}
Any permutation can be expressed as the product of cycles. For instance,
\end{flushleft}





1


5





2


1





3


4





4


3





5


2





?


= (1 5 2)(3 4).





\begin{flushleft}
Denote the flag patterning as a 4-letter string of colors abcd.
\end{flushleft}


\begin{flushleft}
There are two symmetries:
\end{flushleft}


\begin{flushleft}
1. the trivial identity permutation that maps abcd $\rightarrow$ abcd, specifically (a)(b)(c)(d),
\end{flushleft}


?


\begin{flushleft}
2. reflection permutation that maps abcd $\rightarrow$ dcba, specifically
\end{flushleft}





2





\begin{flushleft}
a
\end{flushleft}


\begin{flushleft}
d
\end{flushleft}





\begin{flushleft}
b
\end{flushleft}


\begin{flushleft}
c
\end{flushleft}





\begin{flushleft}
c
\end{flushleft}


\begin{flushleft}
b
\end{flushleft}





\begin{flushleft}
d
\end{flushleft}


\begin{flushleft}
a
\end{flushleft}





?


\begin{flushleft}
, or the cycle product
\end{flushleft}





\begin{flushleft}
\newpage
(a d)(b c).
\end{flushleft}


\begin{flushleft}
The first is a product of four 1-cycles, and the second is the product of two 2-cycles. (An n-cycle is a cycle
\end{flushleft}


\begin{flushleft}
of length n).
\end{flushleft}





\begin{flushleft}
Cycle index polynomial
\end{flushleft}


\begin{flushleft}
P =
\end{flushleft}





\begin{flushleft}
1 · f14 + 1 · f22
\end{flushleft}


2





\begin{flushleft}
We make the following substitution: fn = xn + y n . We use two terms since we are considering two colors.
\end{flushleft}


\begin{flushleft}
P becomes
\end{flushleft}


\begin{flushleft}
P (x, y) =
\end{flushleft}





\begin{flushleft}
(x + y)4 + (x2 + y 2 )2
\end{flushleft}


2





\begin{flushleft}
To find the answer from before, we add the coefficients of this polynomial. This is equivalent to taking P (1, 1),
\end{flushleft}


4


2


\begin{flushleft}
which gives 2 +2
\end{flushleft}


\begin{flushleft}
= 10 , as from before. Importantly, not only is the sum equal, but the constituents of the
\end{flushleft}


2


\begin{flushleft}
sum are similar as well: this is a hint at some sort of combinatorial equivalence between the two processes.
\end{flushleft}





2.3





\begin{flushleft}
P is a Generating Function for Colorings
\end{flushleft}





\begin{flushleft}
Even more remarkably, P is a generating function for each coloring!
\end{flushleft}





\begin{flushleft}
P = x4 + 2x3 y + 4x2 y 2 + 2xy 3 + y 4
\end{flushleft}


\begin{flushleft}
The coefficient of xj y k gives the number of patterns with j squares colored with color 1 and k squares colored
\end{flushleft}


\begin{flushleft}
with color 2.
\end{flushleft}


\begin{flushleft}
Let color 1 be red and color 2 be blue.
\end{flushleft}





\begin{flushleft}
P = x4 + 2x3 y + 4x2 y 2 + 2xy 3 + y 4
\end{flushleft}


\begin{flushleft}
The coefficient of xj y k gives the number of patterns with j squares colored with color 1 and k squares colored
\end{flushleft}


\begin{flushleft}
with color 2.
\end{flushleft}


\begin{flushleft}
Term
\end{flushleft}





\begin{flushleft}
Colorings
\end{flushleft}





\begin{flushleft}
x4
\end{flushleft}


\begin{flushleft}
2x3 y
\end{flushleft}


\begin{flushleft}
4x2 y 2
\end{flushleft}


\begin{flushleft}
2xy 3
\end{flushleft}


\begin{flushleft}
y4
\end{flushleft}





3





\begin{flushleft}
Counting Necklaces: PuMAC 2009
\end{flushleft}





\begin{flushleft}
2009 PUMaC Combinatorics A10 : Taotao wants to buy a bracelet. The bracelets have 7 different beads on
\end{flushleft}


\begin{flushleft}
them, arranged in a circle. Two bracelets are the same if one can be rotated or flipped to get the other. If
\end{flushleft}


3





\begin{flushleft}
\newpage
Figure 1: Symmetries of a 7-gon
\end{flushleft}





\begin{flushleft}
Figure 2: Reflectional symmetry: vertices that get mapped to each other are the same color
\end{flushleft}


\begin{flushleft}
she can choose the colors and placement of the beads, and the beads come in orange, white, and black, how
\end{flushleft}


\begin{flushleft}
many possible bracelets can she buy?
\end{flushleft}


\begin{flushleft}
Solution. Imagine the 7 beads at the vertices of a regular heptagon. See Figure 1.
\end{flushleft}





\begin{flushleft}
7-gon Symmetries
\end{flushleft}


\begin{flushleft}
7 reflections through a vertex and midpoint of opposite side
\end{flushleft}


\begin{flushleft}
7 rotations of n
\end{flushleft}





?


360 ◦


7





\begin{flushleft}
for n $\in$ 1, 2 · · · 7. (n = 7: identity case: 360 degree rotation).
\end{flushleft}





\begin{flushleft}
Permutation Cycle Structure: Reflections
\end{flushleft}


\begin{flushleft}
All the 7 reflections have the same cycle structure, by symmetry. This corresponds to the permutation
\end{flushleft}


\begin{flushleft}
structure (1)(7 2)(6 3)(5 4): see Figure 2.
\end{flushleft}


\begin{flushleft}
In the cycle index polynomial, this is 7f1 f23 , since we have one 1-cycle and three 2-cycles, and 7 such
\end{flushleft}


\begin{flushleft}
reflections, since we can take a reflection through any vertex.
\end{flushleft}





\begin{flushleft}
Permutation Cycle Structure: Rotations
\end{flushleft}


\begin{flushleft}
All the 6 nonidentity rotations are 7-cycles, since 7 is a prime number. This contributes 6 · f7 .
\end{flushleft}


\begin{flushleft}
To understand why this is true, ,we look at a case when n, the number of sides of the polygon, is composite.
\end{flushleft}


\begin{flushleft}
In a hexagon, (6 sides, and 6 is composite) a rotation of 360/3◦ yields (1 3 5)(2 4 6), which would correspond
\end{flushleft}


\begin{flushleft}
to f32 in the cycle index polynomial: see Figure 3.
\end{flushleft}





4





\begin{flushleft}
\newpage
Figure 3: 6 is composite, so we can have disjoint cycles
\end{flushleft}





\begin{flushleft}
The Identity The identity is trivially a product of seven 1-cycles, contribute 1 · f17 .
\end{flushleft}


\begin{flushleft}
Cycle Index Polynomial
\end{flushleft}


\begin{flushleft}
The cycle index polynomial is thus
\end{flushleft}


\begin{flushleft}
7 · f1 f23 + 6 · f7 + 1 · f17
\end{flushleft}


14


\begin{flushleft}
Substituting fn = xn + y n + z n (since we have three colors), this is
\end{flushleft}


\begin{flushleft}
f (x, y, z) =
\end{flushleft}





\begin{flushleft}
7·(x+y+z)(x2 +y 2 +z 2 )3 +6(x7 +y 7 +z 7 )
\end{flushleft}


\begin{flushleft}
+(x+y+z)7
\end{flushleft}





14





\begin{flushleft}
The sum of the coefficients is the desired answer.
\end{flushleft}


\begin{flushleft}
Plugging in 1, we find
\end{flushleft}


\begin{flushleft}
f (1, 1, 1) =
\end{flushleft}





3.1





7 · 34 + 6 · 3 + 3 7


= 198


14





\begin{flushleft}
Multinomial Theorem: Generalizing the Binomial Theorem
\end{flushleft}





\begin{flushleft}
Recall: coefficient of xi y j z k corresponds to the number of necklaces with i of the first color, j of the second
\end{flushleft}


\begin{flushleft}
color, and k of the third.
\end{flushleft}


\begin{flushleft}
(You can replace necklaces with whatever entity you desire, and color with whatever assignment you desire,
\end{flushleft}


\begin{flushleft}
such as chemical radicals.)
\end{flushleft}


\begin{flushleft}
The binomial theorem:
\end{flushleft}


\begin{flushleft}
(x + y)n =
\end{flushleft}





\begin{flushleft}
n ? ?
\end{flushleft}


\begin{flushleft}
X
\end{flushleft}


\begin{flushleft}
X n!
\end{flushleft}


\begin{flushleft}
n k n$-$k
\end{flushleft}


\begin{flushleft}
x y
\end{flushleft}


=


\begin{flushleft}
xi y j
\end{flushleft}


\begin{flushleft}
k
\end{flushleft}


\begin{flushleft}
i!j!
\end{flushleft}


\begin{flushleft}
i+j=n
\end{flushleft}





\begin{flushleft}
k=0
\end{flushleft}





\begin{flushleft}
i,j$\geq$0
\end{flushleft}





\begin{flushleft}
Generalizing,
\end{flushleft}


\begin{flushleft}
(x + y + z)n =
\end{flushleft}





\begin{flushleft}
X
\end{flushleft}


\begin{flushleft}
i+j+k=n
\end{flushleft}


\begin{flushleft}
i,j,k$\geq$0
\end{flushleft}





5





\begin{flushleft}
n! i j k
\end{flushleft}


\begin{flushleft}
xy z
\end{flushleft}


\begin{flushleft}
i!j!k!
\end{flushleft}





\begin{flushleft}
\newpage
X
\end{flushleft}





\begin{flushleft}
(w + x + y + z)n =
\end{flushleft}





\begin{flushleft}
i+j+k+l=n
\end{flushleft}


\begin{flushleft}
i,j,k,l$\geq$0
\end{flushleft}





3.2





\begin{flushleft}
n!
\end{flushleft}


\begin{flushleft}
w i xj y k z l
\end{flushleft}


\begin{flushleft}
i!j!k!l!
\end{flushleft}





\begin{flushleft}
Calculating the number of necklaces of a type: Finding a coefficient of the
\end{flushleft}


\begin{flushleft}
generating function
\end{flushleft}





\begin{flushleft}
Let's say we wanted to find the number of necklaces with 2 red beads, 2 orange beads, and 3 yellow beads:
\end{flushleft}


\begin{flushleft}
x2 y 2 z 3 . We don't want to expand the generating function!
\end{flushleft}





\begin{flushleft}
f (x, y, z) =
\end{flushleft}





\begin{flushleft}
7·(x+y+z)(x2 +y 2 +z 2 )3 +6(x7 +y 7 +z 7 )
\end{flushleft}


\begin{flushleft}
+(x+y+z)7
\end{flushleft}





14





\begin{flushleft}
To obtain x2 y 2 z 3 , in the first term, 7 · (x + y + z)(x2 + y 2 + z 2 )3 we must take a factor of z from (x + y + z),
\end{flushleft}


\begin{flushleft}
and a factor of x2 from the first (x2 + y 2 + z 2 ), y 2 from the second, and z 2 from the third (or any possible
\end{flushleft}


\begin{flushleft}
ordering).
\end{flushleft}


\begin{flushleft}
There are 3! = 6 to pick the ordering of x2 , y 2 and z 2 , so the number of terms from here is 7 · 3! = 42.
\end{flushleft}





\begin{flushleft}
f (x, y, z) =
\end{flushleft}





\begin{flushleft}
7·(x+y+z)(x2 +y 2 +z 2 )3 +6(x7 +y 7 +z 7 )
\end{flushleft}


\begin{flushleft}
+(x+y+z)7
\end{flushleft}





14





\begin{flushleft}
In the middle term, 6(x7 + y 7 + z 7 ), we will not obtain any terms of our form.
\end{flushleft}


\begin{flushleft}
In the last term, (x + y + z)7 we can just apply the multinomial theorem to find that the coefficient is
\end{flushleft}


7!


= 210 .


\begin{flushleft}
nom72, 2, 3 = 2!2!3!
\end{flushleft}


\begin{flushleft}
Thus the coefficient we want is
\end{flushleft}


7 · 3! +





7


2,2,3





?





14





= 18





\begin{flushleft}
There are 18 necklaces with 2 red beads, 2 orange beads, and 3 yellow beads.
\end{flushleft}





3.3





\begin{flushleft}
A Symmetric Generating Function
\end{flushleft}





\begin{flushleft}
f (x, y, z) =
\end{flushleft}





\begin{flushleft}
(x7 +y 7 +z 7 )
\end{flushleft}


\begin{flushleft}
+(x6 y+xy 6 +x6 z+yz 6 +xz 6 +y 6 z)
\end{flushleft}


\begin{flushleft}
+(3x5 y 2 +3x5 z 2 +3y 2 z 5 +3x2 y 5 +3x2 z 5 +3y 5 z 2 )
\end{flushleft}


\begin{flushleft}
+(3x5 yz+3xy 5 z+3xyz 5 )
\end{flushleft}


\begin{flushleft}
+(4x4 y 3 +4y 4 z 3 +4y 3 z 4 +4x4 z 3 +4x3 y 4 +4x3 z 4 )
\end{flushleft}


\begin{flushleft}
+(9x4 y 2 z+9x4 yz 2 +9x2 y 4 z+9x2 yz 4 +9xy 4 z 2 +9xy 2 z 4 )
\end{flushleft}


\begin{flushleft}
+(10x3 y 3 z+10x3 yz 3 +10xy 3 z 3 )
\end{flushleft}


\begin{flushleft}
+(18x3 y 2 z 2 +18x2 y 3 z 2 +18x2 y 2 z 3 )
\end{flushleft}





\begin{flushleft}
Why is the polynomial symmetric?
\end{flushleft}





6





\newpage
4





\begin{flushleft}
Computational Utility
\end{flushleft}





\begin{flushleft}
Just as in problem 1, casework is in principle possible.
\end{flushleft}


\begin{flushleft}
Computuational utility: seen when we increase the number of beads or colors even modestly.
\end{flushleft}


\begin{flushleft}
Suppose we have a necklace with 17 beads and 4 colors.
\end{flushleft}


\begin{flushleft}
P =
\end{flushleft}





\begin{flushleft}
f117 + 16f17 + 17f1 f28
\end{flushleft}


34





\begin{flushleft}
Substituting,
\end{flushleft}


\begin{flushleft}
P =
\end{flushleft}





\begin{flushleft}
(w+x+y+z)17 +16(w17 +x17 +y 17 +z 17 )
\end{flushleft}


\begin{flushleft}
+17(w+x+y+z)(w2 +x2 +y 2 +z 2 )8
\end{flushleft}





34





\begin{flushleft}
P (1, 1, 1, 1) = 5054421344
\end{flushleft}


\begin{flushleft}
You can use the multinomial theorem similarly to find specific coefficients. Number of cases increase fast,
\end{flushleft}


\begin{flushleft}
but only 3 different permutation structures exist, making P\'{o}lya-Burnside easy to apply!
\end{flushleft}





5





\begin{flushleft}
Introduction to Group Theory
\end{flushleft}





\begin{flushleft}
Applications of Abstract Algebra/Group Theory
\end{flushleft}


\begin{flushleft}
$\bullet$ Matrix groups to study the symmetric groups of 3-D solids, various problems in physics, and crystallographic groups.
\end{flushleft}





\begin{flushleft}
$\bullet$ Extension fields for geometrical constructions, including the classical impossibility of duplicating the
\end{flushleft}


\begin{flushleft}
cube, trisecting an angle, and squaring a circle.
\end{flushleft}





\begin{flushleft}
(If n is a positive integer such that the regular n-gon is constructible with ruler and compass, then
\end{flushleft}


\begin{flushleft}
Qk
\end{flushleft}


\begin{flushleft}
m
\end{flushleft}


\begin{flushleft}
n = 2k i=1 pi , where k $\geq$ 0 and the pi are distinct Fermat primes, that is, primes of the form 22 + 1.)
\end{flushleft}


\begin{flushleft}
$\bullet$ Combinatorial enumeration via group action on sets and Burnside 's Lemma (subject of this talk).
\end{flushleft}





\begin{flushleft}
Definition of a Group
\end{flushleft}


\begin{flushleft}
A group (G, ∗) contains a set G of elements and a binary operation ∗.
\end{flushleft}


\begin{flushleft}
$\bullet$ ∗ is closed on G. That is, if g, h $\in$ G, then g ∗ h $\in$ G.
\end{flushleft}


\begin{flushleft}
$\bullet$ ∗ is associative. That is, if a, b, c $\in$ G, then a ∗ (b ∗ c) = (a ∗ b) ∗ c.
\end{flushleft}


\begin{flushleft}
$\bullet$ There exists a (unique) identity element e $\in$ G such that for all g $\in$ G, we have g ∗ e = e ∗ g = g.
\end{flushleft}


\begin{flushleft}
$\bullet$ For all g $\in$ G, there exists an inverse, denoted g $-$1 , such that g ∗ g $-$1 = g $-$1 ∗ g = e
\end{flushleft}


7





\begin{flushleft}
\newpage
Figure 4: Dihedral Group in Odd and Even Cases
\end{flushleft}





\begin{flushleft}
Basic Examples of Groups
\end{flushleft}


\begin{flushleft}
$\bullet$ (C, +), (R, +), (Q, +), (Z, +)
\end{flushleft}


\begin{flushleft}
$\bullet$ The set of symmetries of a rectangle, the Klein 4-group:
\end{flushleft}


\begin{flushleft}
$\bullet$ The group of all permutations of three elements, S3
\end{flushleft}


\begin{flushleft}
$\bullet$ The example most relevant to us: the dihedral group, Dn , the group all symmetries (rotational and
\end{flushleft}


\begin{flushleft}
reflectional) of a regular n-sided polygon, with 2n elements.
\end{flushleft}





\begin{flushleft}
The Dihedral Group, Dn
\end{flushleft}


\begin{flushleft}
The dihedral group with 2n elements consists of the symmetries of an n-gon. We have two cases, n odd,
\end{flushleft}


\begin{flushleft}
and n even.
\end{flushleft}


\begin{flushleft}
When n odd, reflections are through a vertex and the midpoint of the opposite side. When n even, reflections
\end{flushleft}


\begin{flushleft}
are through midpoints of opposite sides.
\end{flushleft}





6


6.1





\begin{flushleft}
Group Action and Burnside's Lemma
\end{flushleft}


\begin{flushleft}
Group Action on Sets
\end{flushleft}





\begin{flushleft}
A group (G, ∗) acts on the set X if there is a function that takes pairs of elements in G and elements in X
\end{flushleft}


\begin{flushleft}
-- (g, x) -- to new elements in X.
\end{flushleft}


\begin{flushleft}
In our case, X will be the set of objects without accounting for symmetry.
\end{flushleft}


\begin{flushleft}
More formally, the group (G, ∗) acts on the set X if there is a function
\end{flushleft}


\begin{flushleft}
f :G×X $\rightarrow$X
\end{flushleft}





\begin{flushleft}
such that when we denote f (g, x) as g(x), we have
\end{flushleft}


\begin{flushleft}
(g1 g2 )(x) = g1 (g2 (x)) for all g1 , g2 $\in$ G, x $\in$ X
\end{flushleft}


\begin{flushleft}
e(x) = x if e is the identity of the group and x $\in$ X
\end{flushleft}





8





\begin{flushleft}
\newpage
The Orbit and Stabilizer
\end{flushleft}


\begin{flushleft}
If G acts on a set X and x $\in$ X, then the stabilizer of x is defined to be the set
\end{flushleft}


\begin{flushleft}
Stab x = \{g $\in$ G | g(x) = x\}
\end{flushleft}


\begin{flushleft}
that is, the set of elements in the group that take the element x to itself.
\end{flushleft}


\begin{flushleft}
Similarly, let Fix g denote the number of elements of X fixed by g, that is the set \{x $\in$ X | g(x) = x\}.
\end{flushleft}


\begin{flushleft}
The set of all outputs of an element x $\in$ X under group action is called the orbit defined as the set
\end{flushleft}


\begin{flushleft}
Orb x = \{g(x) | g $\in$ G\}
\end{flushleft}





\begin{flushleft}
The Orbit-Stabilizer Theorem
\end{flushleft}


\begin{flushleft}
If a finite group G acts on a set X, for each x $\in$ X, we have
\end{flushleft}


\begin{flushleft}
|G| = |Stab x| |Orb x| .
\end{flushleft}


\begin{flushleft}
where |G| denotes the number of elements in the group.
\end{flushleft}


\begin{flushleft}
Intuition: First, recall that there are 24 rotational symmetries of a cube. There are 8 places one vertex can
\end{flushleft}


\begin{flushleft}
go, and 3 places you can put one of its neighbors, yielding 8 · 3 = 24.
\end{flushleft}


\begin{flushleft}
$\bullet$ Fix one face. You can move the cube 4 ways (you can only rotate it). These are the stabilizers.
\end{flushleft}


\begin{flushleft}
$\bullet$ There are 6 faces you can pick. This is the orbit of the face.
\end{flushleft}


\begin{flushleft}
Hence 4 · 6 = 24, the order of the group of cube symmetries, as expected.
\end{flushleft}





6.2





\begin{flushleft}
Burnside's Lemma
\end{flushleft}





\begin{flushleft}
Burnside's Lemma
\end{flushleft}


\begin{flushleft}
If G is a finite group that acts on the elements of a finite set X, and N is the number of orbits of X under
\end{flushleft}


\begin{flushleft}
G, then
\end{flushleft}


\begin{flushleft}
1 X
\end{flushleft}


\begin{flushleft}
N=
\end{flushleft}


\begin{flushleft}
|Fix g|
\end{flushleft}


\begin{flushleft}
|G|
\end{flushleft}


\begin{flushleft}
g$\in$G
\end{flushleft}





\begin{flushleft}
The orbit of an element x $\in$ X refers to all possible colorings you can obtains by some rotation or reflection
\end{flushleft}


\begin{flushleft}
on some coloring.
\end{flushleft}


\begin{flushleft}
If we count the number of orbits, we are counting the number of colorings that are distinct under rotation
\end{flushleft}


\begin{flushleft}
or reflection!
\end{flushleft}





\begin{flushleft}
Proof of Burnside's Lemma
\end{flushleft}


\begin{flushleft}
Consider
\end{flushleft}





\begin{flushleft}
P
\end{flushleft}





\begin{flushleft}
g$\in$G
\end{flushleft}





\begin{flushleft}
|Fix g|.
\end{flushleft}





\begin{flushleft}
But this is also
\end{flushleft}


\begin{flushleft}
|S| =
\end{flushleft}





\begin{flushleft}
X
\end{flushleft}





\begin{flushleft}
|Fix g| =
\end{flushleft}





\begin{flushleft}
g$\in$G
\end{flushleft}





\begin{flushleft}
X
\end{flushleft}


\begin{flushleft}
x$\in$X
\end{flushleft}





9





\begin{flushleft}
|Stab x|
\end{flushleft}





\begin{flushleft}
\newpage
Representative elements from each orbit of X under G, x1 , x2 , · · · , xN .
\end{flushleft}


\begin{flushleft}
If an element x is in the same orbit as xi , then Orb x = Orb xi , and by the orbit-stabilizer theorem,
\end{flushleft}


\begin{flushleft}
|Stab x| = |Stab xi |.
\end{flushleft}


\begin{flushleft}
We have
\end{flushleft}


\begin{flushleft}
X
\end{flushleft}





\begin{flushleft}
|Fix g| =
\end{flushleft}





\begin{flushleft}
N
\end{flushleft}


\begin{flushleft}
X
\end{flushleft}





\begin{flushleft}
X
\end{flushleft}





\begin{flushleft}
|Stab x| =
\end{flushleft}





\begin{flushleft}
i=1 x$\in$Orb xi
\end{flushleft}





\begin{flushleft}
g$\in$G
\end{flushleft}





\begin{flushleft}
N
\end{flushleft}


\begin{flushleft}
X
\end{flushleft}





\begin{flushleft}
|Orb xi | |Stab xi |
\end{flushleft}





\begin{flushleft}
i=1
\end{flushleft}





\begin{flushleft}
Which implies
\end{flushleft}


\begin{flushleft}
X
\end{flushleft}





\begin{flushleft}
|Fix g| =
\end{flushleft}





\begin{flushleft}
N
\end{flushleft}


\begin{flushleft}
X
\end{flushleft}





\begin{flushleft}
|Orb xi | |Stab xi |
\end{flushleft}





\begin{flushleft}
i=1
\end{flushleft}





\begin{flushleft}
g$\in$G
\end{flushleft}





\begin{flushleft}
By the orbit-stabilizer theorem, each of the summands equals |G|.
\end{flushleft}


\begin{flushleft}
Hence
\end{flushleft}


\begin{flushleft}
X
\end{flushleft}





\begin{flushleft}
|Fix g| =
\end{flushleft}





\begin{flushleft}
N
\end{flushleft}


\begin{flushleft}
X
\end{flushleft}





\begin{flushleft}
|Orb xi | |Stab xi | = N · |G|
\end{flushleft}





\begin{flushleft}
i=1
\end{flushleft}





\begin{flushleft}
g$\in$G
\end{flushleft}





\begin{flushleft}
Burnside's Lemma follows:
\end{flushleft}


\begin{flushleft}
N=
\end{flushleft}





\begin{flushleft}
1 X
\end{flushleft}


\begin{flushleft}
|Fix g|
\end{flushleft}


\begin{flushleft}
|G|
\end{flushleft}


\begin{flushleft}
g$\in$G
\end{flushleft}





7





\begin{flushleft}
Intuition: Why P\'{o}lya-Burnside Enumeration Works
\end{flushleft}





\begin{flushleft}
Plugging in 1 Yields Burnside's Lemma!
\end{flushleft}


\begin{flushleft}
Recall the generating functions from the previous examples:
\end{flushleft}


\begin{flushleft}
Problem 1: Number of different flag colorings
\end{flushleft}


\begin{flushleft}
f (x, y) =
\end{flushleft}





\begin{flushleft}
(x + y)4 + (x2 + y 2 )2
\end{flushleft}


2





\begin{flushleft}
24 : number of elements fixed by the identity.
\end{flushleft}


\begin{flushleft}
22 : number of elements fixed by reflection across middle
\end{flushleft}


\begin{flushleft}
2: order of |D2 |.
\end{flushleft}


\begin{flushleft}
Problem 2: Number of different bracelets
\end{flushleft}


\begin{flushleft}
g(x, y, z) =
\end{flushleft}





\begin{flushleft}
7·(x+y+z)(x2 +y 2 +z 2 )3 +6(x7 +y 7 +z 7 )
\end{flushleft}


\begin{flushleft}
+(x+y+z)7
\end{flushleft}





14





\begin{flushleft}
7 · 34 : number of elements fixed by reflections
\end{flushleft}


\begin{flushleft}
6 · 3: number of elements fixed by the six nonidentity rotations
\end{flushleft}


10





\begin{flushleft}
\newpage
37 : number of elements fixed by identity
\end{flushleft}


\begin{flushleft}
Note that plugging in 1 for all the variables gives you N =
\end{flushleft}


\begin{flushleft}
f (1, 1) =
\end{flushleft}





1


\begin{flushleft}
|G|
\end{flushleft}





\begin{flushleft}
P
\end{flushleft}





\begin{flushleft}
g$\in$G
\end{flushleft}





\begin{flushleft}
|Fix g|!
\end{flushleft}





24 + 22


7 · 34 + 6 · 3 + 37


\begin{flushleft}
; g(1, 1) =
\end{flushleft}


2


14





\begin{flushleft}
Why the Generating Function Substitution Works
\end{flushleft}


\begin{flushleft}
Recall:
\end{flushleft}


\begin{flushleft}
If an object that can be colored with k colors has a symmetry as follows: A permutation with p1 cycles of
\end{flushleft}


\begin{flushleft}
length 1, p2 of length 2, pn of length n (pi = 0 allowed) contributes
\end{flushleft}


\begin{flushleft}
f1p1 f2p2 · · · fnpn
\end{flushleft}


\begin{flushleft}
to the cycle index polynomial. If you have k colors, substitute fi = (ci1 + ci2 + · · · + cik ).
\end{flushleft}


\begin{flushleft}
Intuition. To count fixed elements, (|Fix g|), all entities in the respective cycles must be the same color!
\end{flushleft}


\begin{flushleft}
In the generating function, ci1 cj2 ck3 represents i, j, k instances of c1 , c2 , c3 respectively.
\end{flushleft}


\begin{flushleft}
To be the same, we can have substitute cpnn for any color, since it doesn't matter what color we pick, so we
\end{flushleft}


\begin{flushleft}
substitute fi = (ci1 + ci2 + · · · + cik ).
\end{flushleft}





7.1





\begin{flushleft}
Examples: Why the Substitution Works
\end{flushleft}





\begin{flushleft}
Reflection in the four color flag example.
\end{flushleft}


\begin{flushleft}
Let flag patterning by 4-letter string of colors abcd.
\end{flushleft}


\begin{flushleft}
Permutation structure: abcd $\rightarrow$ dcba: (a d)(b c)
\end{flushleft}


\begin{flushleft}
Term in cycle index polynomial: f22 .
\end{flushleft}


\begin{flushleft}
Need to substitute f2 = x2 + y 2 .
\end{flushleft}


\begin{flushleft}
For a pattern to be fixed, a = d and b = c, as in
\end{flushleft}


\begin{flushleft}
If xi y j represents i, j instances of colors x, y, then x2 + y 2 provide two terms: we can either have both parts
\end{flushleft}


\begin{flushleft}
of the cycle be red (x), or blue (y).
\end{flushleft}


\begin{flushleft}
Rotation of a Hexagon: D6
\end{flushleft}


\begin{flushleft}
In a hexagon, a rotation of 360/3◦ yields (1 3 5)(2 4 6)
\end{flushleft}


\begin{flushleft}
In this case, we must have 1 = 3, 2 = 4, and 5 = 6.
\end{flushleft}


\begin{flushleft}
Hence the cycle index term is f32 .
\end{flushleft}


\begin{flushleft}
If we have two colors, substitute f32 = (x3 + y 3 )2 to give all possible colorings.
\end{flushleft}


11





\begin{flushleft}
\newpage
Figure 5: 6 is composite, so we can have disjoint cycles
\end{flushleft}





\begin{flushleft}
Figure 6: Reflectional symmetry: vertices that get mapped to each other are the same color
\end{flushleft}


\begin{flushleft}
Reflection of a Heptagon: D7
\end{flushleft}


\begin{flushleft}
Suppose we have a heptagon.
\end{flushleft}


\begin{flushleft}
Notice that the permutation maps (1)(7 2)(6 3)(5 4).
\end{flushleft}


\begin{flushleft}
This corresponds to the cycle structure f23 f1 .
\end{flushleft}


\begin{flushleft}
We substitute fn = xn + y n giving f23 f1 = (x2 + y 2 )3 (x + y).
\end{flushleft}


\begin{flushleft}
The reds, oranges, and greens have to be equal, so we must have either x2 or y 2 three times (so we cube).
\end{flushleft}


\begin{flushleft}
Finally, the blue can be either color, x or y.
\end{flushleft}





8





\begin{flushleft}
Chemical Isomer Enumeration
\end{flushleft}





\begin{flushleft}
How many different chemical compounds can be made by attaching H, CH3 , or OH radicals to each of the
\end{flushleft}


\begin{flushleft}
carbon atoms in the benzene ring pictured below?
\end{flushleft}





12





\begin{flushleft}
\newpage
C
\end{flushleft}





\begin{flushleft}
C
\end{flushleft}





\begin{flushleft}
C
\end{flushleft}





\begin{flushleft}
C
\end{flushleft}





\begin{flushleft}
C
\end{flushleft}





\begin{flushleft}
C
\end{flushleft}





\begin{flushleft}
With your help, if time and volunteers!
\end{flushleft}





9





\begin{flushleft}
Additional Problems
\end{flushleft}





\begin{flushleft}
With the exception of \#1, these are an assortment of problems in which it isn't immediately clear that
\end{flushleft}


\begin{flushleft}
Burnside's Lemma can be applied.
\end{flushleft}


\begin{flushleft}
Source: Art of Problem Solving Forums
\end{flushleft}


\begin{flushleft}
1. Two of the squares of a 7 × 7 checkerboard are painted yellow, and the rest are painted green. Two
\end{flushleft}


\begin{flushleft}
color schemes are equivalent if one can be obtained from the other by applying a rotation in the plane
\end{flushleft}


\begin{flushleft}
of the board. How many inequivalent color schemes are possible? (AIME 1996, \#7)
\end{flushleft}


\begin{flushleft}
2. Find the number of second-degree polynomials f (x) with integer coefficients and integer zeros for which
\end{flushleft}


\begin{flushleft}
f (0) = 2010. (AIME 2010, \#10)
\end{flushleft}


\begin{flushleft}
3. Two quadrilaterals are considered the same if one can be obtained from the other by a rotation and a
\end{flushleft}


\begin{flushleft}
translation. How many different convex cyclic quadrilaterals are there with integer sides and perimeter
\end{flushleft}


\begin{flushleft}
equal to 32? (AMC 12A 2010, \#25)
\end{flushleft}


\begin{flushleft}
4. How many subsets \{x, y, z, t\} $\subset$ N are there that satisfy the following conditions?
\end{flushleft}





\begin{flushleft}
12 $\leq$ x $<$ y $<$ z $<$ t
\end{flushleft}


\begin{flushleft}
x + y + z + t = 2011
\end{flushleft}


\begin{flushleft}
5. Prove that, for all positive integers n and k, we have
\end{flushleft}


\begin{flushleft}
n|
\end{flushleft}





\begin{flushleft}
n$-$1
\end{flushleft}


\begin{flushleft}
X
\end{flushleft}


\begin{flushleft}
i=0
\end{flushleft}





\begin{flushleft}
where a|b means that a divides b.
\end{flushleft}





13





\begin{flushleft}
k gcd(i,n)
\end{flushleft}





\newpage



\end{document}
